
\TeX\ Book Exercises

{\bf Exercise 4.1 }

Using groupings, bold font, slants, and special characters and en dashes

Ulrich Dieter, {\sl Journal f\"ur des reind und angewundte Mathematik\/} {\bf201} (1959), 37--50

{\bf Exercise 4.2 |

{\sl Explain how to typeset a} roman {\sl word in the middle of an italicized sentence}.

Using Fonts ...
\font\ninerm=cmr9
\font\eightrm=cmr8
\font\sevenrm=cmr7
\font\sixrm=cmr6
\font\fiverm=cmr5
\font\twelverm=cmr12
A sample is \tenrm smaller \ninerm and smaller and \eightrm
and smaller \sevenrm and smaller \sixrm and smaller \fiverm and smaller \tenrm



\twelverm Let's compare twelve point  ABCDEFG with \fiverm five ABCDEFG


\twelverm A\fiverm A \twelverm B\fiverm B

% remember to switch back to normal font 
\tenrm


{\bf Chapter 5}


{\it Exercise 5.1}

How to make 'shelfful' not have a ligature let's try shelf{}ful.

{\it Exercise 5.2}

How get three spaces in a row with using a\ \ \ b or a{} {} {} b

{\it Exercise 5.3}

\centerline {This information should be {centered}.}
\centerline {so should this}
\centerline {This information should be {\it centered}.}

There is a bit at the end of this chapter on defining new control sequences.  Will have to come back to this, when needed.


{\bf Chapter 6}



\end

